\chapter*{Introdução}
\addcontentsline{toc}{chapter}{Introdução}
\pagenumbering{arabic}

O setor de telecomunicações com o passar do tempo sempre demandou de melhorias em sua rede para poder atender às necessidades de seus usuários. Com isso de tempos em tempos novas tecnologias são criadas para poder atender a essa demanda, e com essas novas tecnologias novos \textit{hardwares} são desenvolvidos da mesma forma. Geralmente quando novas tecnologias são criadas no setor de transmissão \textit{wireless}, elas buscam uma maior largura de banda e, as vezes, as utilização de novas faixas do espectro de frequências.

Em todo e qualquer sistema \textit{wireless}, como \textit{wifi, bluetooth}, telefonia móvel entre outros, o uso de antenas é indispensável uma vez que esta é a responsável por enviar e receber os dados em um sistema de comunicação. Toda e qualquer antena é desenvolvida visando atender a determinadas características do sistema em que serão usadas, dentre elas estão a frequência de operação e a largura de banda, além disso também devem atender às características físicas do dispositivo onde a antena será instalada, o que para dispositivos móveis geralmente é um espaço reduzido. Para atender a essas necessidades uma alternativa bastante utilizada são as antenas \textit{patch}.%colocar alguama referencia aqui.

As antenas \textit{patch}, ou planares, aparecem como uma boa alternativa para dispositivos móveis por ocuparem pouco espaço, serem leves, possuírem uma boa largura de banda, podem operar em diferentes bandas (\textit{dual-band}) e em frequências mais altas. Tais antenas têm sido muito utilizadas em telefonias móveis, que constantemente passam por melhorias e constantemente necessitam de novas antenas que atendam às especificações de cada tecnologia.Cada projetista utiliza uma estratégia ao definir a antena a ser utilizada, que pode ser a utilização de um modelo clássico de antena e o modifica segunda as necessidades do projeto, ou propõe um novo modelo. Independente da alternativa utilizada a antena projetada precisa estar casada com o sistema com o qual será utilizada, o que muitas vezes não é possível, tornando necessária a utilização de circuitos de casamento. Uma maneira para se verificar a impedância de sistemas é conhecida com TDR. % colocar uma referencia aqui

TDR é um método bastante utilizado para a aquisição das características físicas dos sistemas medidos no domínio do positivo tempo. Essa técnica possibilita a aquisição de dados como: tamanho do dist, impedância, capacitância, indutância e localização pontos de descontinuidade. Recentemente, tem sido utilizada para a caracterização de antenas, como uma alternativa mais viável que as medições utilizando VNA, que necessitam da estrutura de uma câmara anecoica para a realização de medidas.

\section*{Objetivos}

O objetivo principal do trabalho é a utilização do método TDR para a caracterização de estruturas planares, e mostrar a partir da resposta obtida que método é passível de ser utilizado como uma forma alternativa na avaliação de estruturas.  Além da caracterização de estruturas, será mostrado que o método também pode ser usado em testes de validação, como, por exemplo, na avaliação de possíveis possíveis falhas que tenham ocorrido durante ao processo de produção. Além disso objetiva-se também, fornecer de forma simples uma coleção de estruturas com seus respectivos sinais de resposta, afim de auxiliar na análise de uma estrutura mais complexa.

\section*{Organização do Trabalho}

Este trabalho está organizado em 5 capítulos da seguinte maneira: o Capítulo 1 apresenta  umas introdução teórica sobre estruturas planares, metodologias de projeto e suas principais aplicações. O Capítulo 2 trás uma introdução sobre o método TDR, a forma como o método trabalha, quais são as respostas típicas para determinados circuitos, além do conhecimento matemático sobre o método desenvolvido até o momento. o Capítulo 3 faz referências a algumas fomas de assinaturas existentes, e como elas podem ser utilizadas. O Capítulo 4 ilustra a metodologia utilizada para o desenvolvimento do trabalho trabalho. Por fim, o Capítulo 5 apresenta os resultados obtidos no trabalho.

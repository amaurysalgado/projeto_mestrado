\chapter*{Conclusões Preliminares}
\addcontentsline{toc}{chapter}{Conclusões Preliminares}
O método de medição TDR, mostra-se com uma ferramenta de grande valia quando se trada da avaliação de sistemas, possibilitando a sua utilização em diversas áreas de aplicação, tanto em telecomunicações quanto em monitoramentos. Essa característica se dá pela relativa simplicidade de medição e interpretação dos dados por ela mostrados. 

Nessa avaliação inicial está sendo levando em consideração apenas o aspecto quantitativo de análise. O TDR mostrou-se de grande valia por conseguir distinguir diferentes partes da antena, assim como a possibilidade de poder fazer uma classificação do tipo de respostas que cada tipo de antena pode mostrar, dessa forma facilitando de identificação de estruturas utilizadas da confecção de cada antena e definir sua assinatura elétrica. Com esses resultados iniciais também foi possível fazer a identificação de alguns problemas que podem ocorrer e que afetam a medição para o projeto proposto, como foi o caso do plano de terra, que por ele não possuir um comprimento maior ou igual ao comprimento da antena em questão, não foi possível fazer a caracterização completa da antena.

\section*{Próximas Etapas}
\addcontentsline{toc}{section}{Próximas Etapas}
Para o decorrer da pesquisa está previsto a realização de novas medições em estruturas diferentes, assim como também o equacionamento do padrão de resposta de algumas estruturas, como por exemplo a estrutura de meandro, apresentada nos resultados parciais. Serão investigadas também outras técnicas tais como: TDR diferencial e FDR (Frequency Domain Reflectometry).

Estamos avaliando a possibilidade de caracterização de segmentos planares através da \textbf{matriz-Z}, em seguimentos retangulares a princípio, ou por outros métodos como o \textbf{método das cavidades} ou \textbf{método dos momentos}(\cite{Gupta}).

Estruturas mais complexas, como módulos de transmissão e recepção (T-R modules) e estruturas de array planar, também poderão ser analisadas.


%\cite{McCarb}, \cite{Bhartia} , \cite{Cataldo}
